\documentclass[11pt]{article}
\usepackage[T1]{fontenc}
\usepackage[a4paper,margin=1in]{geometry}
\usepackage{enumitem}
\usepackage{array}
\usepackage{hyperref}
\usepackage{amsfonts}
\usepackage{amsmath}
\usepackage[style=numeric,backend=biber]{biblatex}
\addbibresource{references.bib}

\title{Real Number Definitions, Equivalence Classes, and Embeddings (No Countable Choice)}
\author{}
\date{}

\begin{document}
\maketitle

This document lists the first twenty-three constructive definitions of real numbers, partitions them into equivalence classes, and records the canonical embeddings. Note that while many equivalences hold in standard constructive mathematics (assuming Countable Choice, \(\mathsf{AC}_\omega\)), in plain Cubical Agda without \(\mathsf{AC}_\omega\), the hierarchy is more fractured (e.g., Cauchy reals are not provably Dedekind reals). General surveys and formalization resources include \cite{RichmanSurvey,TroelstraVanDalen1988,AgdaWiki}.

\section*{1. Definitions (1--33)}
\begin{enumerate}[leftmargin=*]
  \item $\mathbb{R}_\mathrm{D}$: Dedekind reals (located cuts of $\mathbb{Q}$) \cite{Dedekind1872,Bishop1967,BishopBridges1985,BridgesRichman1987,LubarskyRathjen2007,TomasiDedekind,Zanardo2012,MathOverflowLocales,Stout1976,nLabRealNumbers,Hall2010,Devillanova2021}.
  \item $\mathbb{R}_\mathrm{C}$: Cauchy reals (modulated Cauchy sequences of rationals, quotiented) \cite{Bishop1967,BishopBridges1985,LubarskyENTCS2007,Murray2022,MathOverflowDedekindCauchy,MathOverflowCauchyModulus,nLabCauchyReals}.
  \item $\mathbb{R}_\mathrm{E}$: Eudoxus reals (almost-homomorphisms $\mathbb{Z} \to \mathbb{Z}$) \cite{Arthan2004,Schanuel1992,Street2003,PROMYS2023,Fokma2021,Keskin2025,MathOverflowBauerHanson}.
  \item $\mathbb{R}_\mathrm{FC}$ / $\mathbb{R}_\mathrm{I}$: fast Cauchy reals / interval reals (Cauchy sequences with explicit moduli or nested rational intervals) \cite{Chernov2021,AbrialCansell2021,Weihrauch2000,WikiNestedIntervals}.
  \item $\mathbb{R}_\mathrm{CF}$: continued fraction reals (streams of partial quotients) \cite{BrilliantCF,WikiContinuedFraction,ConstructiveCF}.
  \item $\mathbb{R}_{b}$: coinductive base-$b$ reals (digit streams, e.g., binary/decimal) \cite{WiesnetKopp2022}.
  \item $\mathbb{R}_\mathrm{SD}$: signed-digit reals (streams over $\{-1,0,1\}$) \cite{Ambridge2024,TypeTopology,WiesnetKopp2022,WikiSignedDigit}.
  \item $\mathbb{R}_\mathrm{ID}$: interval domain reals (maximal elements of the interval domain) \cite{vanDerWeideFrumin2024,PattinsonMohammadian2021,deJong2023,BauerIntervalDomain,BauerTaylor2009,EdalatHeckmann1998,EdalatRealComputability,DiGianantonioReals,TaylorASDReals,Escardo1996,BauerKavkler2009}.
  \item $\mathbb{R}_\mathrm{L}$: lower reals (rounded lower sets of $\mathbb{Q}$) \cite{nLabOneSided,BlechschmidtHutzler2019,CoqConstructiveReals}.
  \item $\mathbb{R}_\mathrm{U}$: upper reals (rounded upper sets of $\mathbb{Q}$) \cite{nLabOneSided,BlechschmidtHutzler2019,CoqConstructiveReals}.
  \item $\mathbb{R}_\mathrm{M}$: MacNeille reals (double-negation closed cuts) \cite{MacNeille1937,nLabMacNeille,FourmanGrayson1982,Shulman2022}.
  \item $\mathbb{R}_\mathrm{H}$: HIT/HoTT-book reals (higher inductive type with universal property) \cite{HoTT2013,nLabHoTTReal,Booij2017,Booij2020,BauerHoTTReals2016,PrataliThesis}.
  \item $\mathbb{R}_\mathrm{ES}$: Escard\'o--Simpson reals (least Cauchy-complete subobject of $\mathbb{R}_\mathrm{D}$ containing $\mathbb{Q}$) \cite{EscardoSimpson2001,Booij2020}.
  \item $\mathbb{R}_\mathrm{formal}$: formal/locale reals (points of the locale of reals) \cite{MacLaneMoerdijk1992,Johnstone1982,Johnstone2002,PicadoPultr2012,Grossack2024,nLabRealNumbers}.
  \item $\mathbb{R}_\mathrm{init}$: initial sequentially modulated Cauchy-complete Archimedean ordered field \cite{EscardoSimpson2001,Booij2020,nLabArchimedean}.
  \item $\mathbb{R}_\mathrm{term}$: terminal Archimedean ordered field \cite{EscardoSimpson2001,MacLaneMoerdijk1992,nLabArchimedean}. Items 34 and 35 are distinct definitions with different construction proofs, but they result in the same object if the category is well-behaved.
  \item $\mathbb{R}_\mathrm{DedComp}$: Dedekind-complete ordered field (axiomatic characterization) \cite{MacLaneMoerdijk1992,Johnstone2002,BauerCompleteOrderedFields}.
  \item $\mathbb{R}_\mathrm{CauComp}$: Cauchy-complete ordered field (axiomatic characterization of the Cauchy completion) \cite{BishopBridges1985,TroelstraVanDalen1988}.
  \item $\mathbb{R}_\mathrm{Tarski}$: Archimedean Tarski group reals (characterization via Tarski's axioms) \cite{Tarski1951,Devillanova2021}.
  \item $[0,1]_\mathrm{coalg}$: unit interval as a terminal coalgebra \cite{EscardoSimpson2001,AdamekMiliusMoss2025,nLabCoalgebraInterval,Dusko2002}.
  \item $\mathbb{R}^+_\mathrm{coalg}$: positive reals as a terminal coalgebra \cite{EscardoSimpson2001,AdamekMiliusMoss2025,nLabCoalgebraInterval}.
  \item Sheaf-theoretic reals: the internal real numbers object in a topos \cite{MacLaneMoerdijk1992,Johnstone2002,Stout1976,Grossack2024,nLabRealNumbers}.
  \item Real numbers object (RNO) in a topos \cite{MacLaneMoerdijk1992,Johnstone2002,Stout1976,Grossack2024,nLabRealNumbers}. Items 41 and 42 are essentially the same mathematical object described from two different points of view (internal language vs.\ category theory).
  \item $\mathbb{R}_\mathrm{SDG}$: Smooth Reals (synthetic differential geometry). In Synthetic Differential Geometry, the reals are defined to include ``nilpotent'' infinitesimals (elements \(d\) where \(d^2 = 0\) but \(d \neq 0\)). These are distinct from standard Dedekind/Cauchy reals because they violate the field axiom \(x \neq 0 \implies x\) is invertible (nilpotents are not invertible). They are a distinct mathematical object internal to a smooth topos, not isomorphic to the usual Cauchy/Dedekind reals \cite{Kock2006,KosteckiSDG,MathOverflowSDG,nLabSmoothTopos}.
  \item $^*\mathbb{R}$: Hyperreals (non-standard analysis). These include infinite and infinitesimal numbers. While usually constructed classically (using ultrafilters), there are constructive approaches (e.g., Palmgren's constructive non-standard analysis) that result in a structure distinct from \(\mathbb{R}_\mathrm{D}\) or \(\mathbb{R}_\mathrm{C}\). They are strict extensions of the ordinary reals \cite{Robinson1966,Keisler1976,PalmgrenCNSA}.
  \item Predicative Reals: In systems stricter than Agda (like those prohibiting impredicativity), Dedekind cuts must be restricted (e.g., to ``generalized'' or ``weak'' cuts) to avoid circular definitions. The document hints at this with lower/upper reals (items 9, 10), but specific predicative formalizations often stand alone \cite{FefermanPredicative,BridgesMaietti2006,PaulinMohring1993}.
  \item $\mathbf{No}$: Surreal Numbers (Conway's construction). While the Surreals contain the Reals, the ``Real subset'' of the Surreals is a valid constructive definition of the reals. Inside \(\mathbf{No}\) there is a canonical embedded copy of \(\mathbb{R}\); this embedding can be used as yet another definition of the real line \cite{Conway1976,Gonshor1986,Mamane2004,Knuth1974,WikiSurreal}.
  \item Geometric Reals: Defined synthetically in Euclidean Geometry (e.g., Tarski's axioms for geometry, or Hilbert's axioms). Defined as ``points on a line'' rather than arithmetically. Constructively, relating ``points on a line'' to ``Dedekind cuts'' is a non-trivial project (requires the Cantor-Dedekind axiom) \cite{Tarski1951,Tarski1959,Beeson2015,Perout2013,WikiCantorDedekind}.
  \item Computable Reals (Turing): Specifically defined as ``Turing machines that output digits''. This is distinct from \(\mathbb{R}_\mathrm{C}\) because \(\mathbb{R}_\mathrm{C}\) allows \emph{any} function, whereas Computable Reals restrict the functions to computable ones. In strongly normalizing type theories, every \emph{definable} function is computable (meta-theoretically), so formalising computable reals inside such a system is natural. But this does not by itself make \(\mathbb{R}_\mathrm{C}\) ``the same'' as the usual Cauchy reals object; you still have to choose a semantic setting (e.g. an effective topos) where every function in the space is interpreted computably \cite{Turing1937,Aberth1980,Weihrauch2000,Geuvers2000}.
  \item Decimal / Base-10 Cauchy Reals: Reals as equivalence classes of decimal expansions; classically standard, but constructively they are just another representation type akin to digit-based reals \cite{Tao2016,BartleSherbert2011,WikiRealConstruction}.
  \item Apartness / Located Reals (Bishop Style): Reals as located, rounded lower cuts (or Cauchy sequences with an apartness relation). Bishop's ``Constructive Analysis'' uses a specific flavor of Cauchy reals (regular sequences with a fixed modulus of convergence, usually \(1/n\)). While often isomorphic to standard Cauchy reals, in a strict intensional type theory, the specific choice of modulus makes the type definition distinct \cite{Bishop1967,BishopBridges1985,BridgesMaietti2006,ResearchGateBishopReals}.
  \item Filter-based Completions: Reals as equivalence classes of Cauchy filters (or regular Cauchy filters) on~\(\mathbb{Q}\); conceptually close to Dedekind/Cauchy but more topological \cite{FarahWeiss2015}.
  \item Locale-of-Reals Variants: Several flavours exist internally: lower reals, upper reals, rounded reals, etc. Some topos texts distinguish a few different ``real objects'' as default~\cite{Vickers1996,FourmanGrayson1982,Johnstone1982,nLabRealNumbers}.
\end{enumerate}

\section*{2. Equivalence Classes (Provable without Countable Choice)}
Each class below consists of definitions that are often equivalent in constructive mathematics with \(\mathsf{AC}_\omega\). In plain Cubical Agda, equivalences between classes (e.g., A and B) may fail.

\subsection*{A. Dedekind-Type Completions}
$\mathbb{R}_\mathrm{D}$, $\mathbb{R}_\mathrm{formal}$, Sheaf-theoretic reals, and the real numbers object in a topos all present the Dedekind completion of $\mathbb{Q}$ via localized/topos-theoretic perspectives \cite{MacLaneMoerdijk1992,Johnstone1982,Johnstone2002,nLabRealNumbers,Grossack2024}.

\subsection*{B. Cauchy/HIT-Type Completions}
$\mathbb{R}_\mathrm{C}$, $\mathbb{R}_\mathrm{FC}$, $\mathbb{R}_\mathrm{I}$, $\mathbb{R}_\mathrm{H}$, $\mathbb{R}_\mathrm{init}$, $\mathbb{R}_\mathrm{ES}$, and (axiomatically) $\mathbb{R}_\mathrm{CauComp}$ represent the Cauchy completion, differing only in presentation (explicit modulus, higher inductive, universal property, or internal closure of $\mathbb{Q}$ within $\mathbb{R}_\mathrm{D}$) \cite{HoTT2013,Booij2017,Booij2020,Murray2022}.

\subsection*{C. Representation (Digit/Continued Fraction) Pre-Reals}
\(\mathbb{R}_\mathrm{CF}\), \(\mathbb{R}_{b}\), \(\mathbb{R}_\mathrm{SD}\), Decimal/Base-10 reals give concrete digit- or fraction-based streams. These are not literally ``the reals'' until quotiented; they are ``presentations of \(\mathbb{R}\)''. Raw types are not fields because of non-unique encodings, but their quotients by the appropriate equivalence relation coincide with Class B~\cite{Weihrauch2000,WiesnetKopp2022,MathStackConstructiveReals,BergerHou2007}.

\subsection*{D. Coalgebraic Subspaces}
$[0,1]_\mathrm{coalg}$ and $\mathbb{R}^+_\mathrm{coalg}$ describe the unit interval and positive reals as terminal coalgebras. Constructively they model subspaces of $\mathbb{R}_\mathrm{D}$ but do not deliver the entire field without additional principles \cite{EscardoSimpson2001,AdamekMiliusMoss2025,nLabCoalgebraInterval,Rutten2000}.

\subsection*{E. Generalized Cuts}
\(\mathbb{R}_\mathrm{L}\), \(\mathbb{R}_\mathrm{U}\), and \(\mathbb{R}_\mathrm{M}\) relax locatedness/density requirements. They contain \(\mathbb{R}\) as a canonical subobject (or as maximal elements) but are bigger structures and not isomorphic to \(\mathbb{R}\) as an ordered field~\cite{Vickers1996,MacNeille1937,BlechschmidtHutzler2019,FourmanGrayson1982,Shulman2022}.

\subsection*{F. Domain-Theoretic}
$\mathbb{R}_\mathrm{ID}$ sits in domain/locale theory. Its equivalence to Dedekind reals is not provable in plain Cubical Agda, so it remains a separate class. Note that in frameworks like Abstract Stone Duality or general Topos Theory, these often collapse into Class A (Dedekind-type) via duality results, but internally to Agda without extra axioms, the distinction is maintained \cite{AbramskyJung1994,EdalatHeckmann1998,vanDerWeideFrumin2024,PattinsonMohammadian2021,BauerIntervalDomain,DiGianantonioReals,TaylorASDReals}.

\subsection*{G. Axiomatic/Universal Characterizations}
\(\mathbb{R}_\mathrm{term}\), \(\mathbb{R}_\mathrm{DedComp}\), and \(\mathbb{R}_\mathrm{Tarski}\) capture Dedekind-like structures via universal properties or axioms. These are abstract characterizations; one still needs to show they are realized by some concrete construction. They coincide with the usual reals only once classical principles (e.g., choice) are assumed~\cite{MacLaneMoerdijk1992,Johnstone2002,BauerCompleteOrderedFields}.

\subsection*{H. Isolated/Unresolved}
$\mathbb{R}_\mathrm{E}$ (Eudoxus reals) currently lacks a constructive proof of equivalence with either Dedekind or Cauchy completions. We therefore mark it as isolated \cite{Arthan2004,Schanuel1992,PROMYS2023,Fokma2021,Keskin2025}.

\section*{3. Canonical Embeddings (No Countable Choice)}
\begin{itemize}[leftmargin=*]
  \item Cauchy into Dedekind: there is a canonical field embedding $\mathbb{R}_\mathrm{C} \hookrightarrow \mathbb{R}_\mathrm{D}$ that sends each Cauchy real to the located cut defined by its values \cite{BishopBridges1985,LubarskyENTCS2007,Lubarsky2007,MathOverflowDedekindCauchy}.
  \item Cauchy/HIT equivalences: $\mathbb{R}_\mathrm{FC}$, $\mathbb{R}_\mathrm{I}$, $\mathbb{R}_\mathrm{H}$, $\mathbb{R}_\mathrm{init}$, $\mathbb{R}_\mathrm{ES}$, and $\mathbb{R}_\mathrm{CauComp}$ are inter-definable and embed into $\mathbb{R}_\mathrm{C}$ (hence into $\mathbb{R}_\mathrm{D}$) \cite{HoTT2013,Booij2017,Booij2020,EscardoSimpson2001}.
  \item Representation quotients: the quotient of $\mathbb{R}_\mathrm{CF}$, $\mathbb{R}_{b}$, or $\mathbb{R}_\mathrm{SD}$ by the digit-equivalence relation embeds into $\mathbb{R}_\mathrm{C}$. Without quotienting there is a surjection obstruction due to non-unique encodings \cite{Weihrauch2000,BergerHou2007,MathStackConstructiveReals,WiesnetKopp2022}.
  \item Dedekind to generalized cuts: taking lower (resp. upper) shadows gives embeddings $\mathbb{R}_\mathrm{D} \to \mathbb{R}_\mathrm{L}$ and $\mathbb{R}_\mathrm{D} \to \mathbb{R}_\mathrm{U}$. Composing with double-negation closure embeds into $\mathbb{R}_\mathrm{M}$ \cite{Vickers1996,nLabOneSided,nLabMacNeille,MacNeille1937}.
  \item Coalgebraic subspaces: maps $[0,1]_\mathrm{coalg} \to [0,1] \subseteq \mathbb{R}_\mathrm{D}$ and $\mathbb{R}^+_\mathrm{coalg} \to \mathbb{R}^+ \subseteq \mathbb{R}_\mathrm{D}$ exist, but surjectivity constructs require additional principles \cite{EscardoSimpson2001,AdamekMiliusMoss2025,nLabCoalgebraInterval}.
  \item Axiomatic to Dedekind/Cauchy: the objects $\mathbb{R}_\mathrm{term}$, $\mathbb{R}_\mathrm{DedComp}$, and $\mathbb{R}_\mathrm{Tarski}$ admit maps into $\mathbb{R}_\mathrm{D}$ matching their universal properties, yet converses rely on classical axioms and are not derivable constructively \cite{MacLaneMoerdijk1992,Johnstone2002,BauerCompleteOrderedFields,Devillanova2021}.
  \item Domain to Dedekind: $\mathbb{R}_\mathrm{ID}$ maps to $\mathbb{R}_\mathrm{D}$ via evaluation at maximal elements, but equivalence is unproven constructively \cite{AbramskyJung1994,EdalatHeckmann1998,BauerIntervalDomain,EdalatRealComputability}.
  \item Eudoxus: natural maps from $\mathbb{R}_\mathrm{E}$ into either $\mathbb{R}_\mathrm{C}$ or $\mathbb{R}_\mathrm{D}$ require countable choice for surjectivity and therefore remain dashed (non-provable) in this setting \cite{Arthan2004,PROMYS2023,Fokma2021,Keskin2025,MathOverflowBauerHanson}.
\end{itemize}

\printbibliography

\end{document}
